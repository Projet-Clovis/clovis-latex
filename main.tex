\documentclass{article}


% -------------------- Packages --------------------
\usepackage[utf8]{inputenc}
\usepackage[T1]{fontenc}%
\usepackage{lmodern}%
\usepackage{textcomp}%
\usepackage{lastpage}%


\usepackage{amsmath}
\usepackage{amssymb}
\usepackage{graphicx}
\usepackage{float}

\usepackage{array}
\usepackage{arydshln}


\usepackage{soulutf8} % highlighting
\usepackage{inconsolata} % monospace font
\usepackage{minted}
\usepackage{fontawesome5}


\usepackage{xcolor}
\usepackage[framemethod=tikz]{mdframed}
\usepackage{tikzpagenodes}
\usetikzlibrary{calc}


\usepackage{hyperref}
\hypersetup{
    colorlinks=true,
    linkcolor=blue,
    filecolor=magenta,
    urlcolor=cyan,
    pdftitle={},
    pdfpagemode=FullScreen,
}

\urlstyle{same}

\usepackage{bookmark}
\hypersetup{hidelinks}


% -------------------- Colors --------------------
\definecolor{gray300}{HTML}{d1d5db}
\definecolor{gray600}{HTML}{4b5563}
\definecolor{gray700}{HTML}{374151}
\definecolor{gray800}{HTML}{1f2937}

\definecolor{definition}{HTML}{2f80ed}
\definecolor{definition-bg}{HTML}{dfebfc}

\definecolor{example-color}{HTML}{6b7280}
\definecolor{example-bg-color}{HTML}{eceef2}

\definecolor{byheart-color}{HTML}{ffb0da}
\definecolor{byheart-bg-color}{HTML}{fff3f9}

\definecolor{danger-color}{HTML}{e6505f}
\definecolor{danger-bg-color}{HTML}{fbe4e7}

\definecolor{summary-color}{HTML}{e8ae95}
\definecolor{summary-bg-color}{HTML}{fbf2ef}

\definecolor{reminder-color}{HTML}{7689e8}
\definecolor{reminder-bg-color}{HTML}{eaedfb}

\definecolor{advice-color}{HTML}{685ea1}
\definecolor{advice-bg-color}{HTML}{e8e6f0}

\definecolor{remark-color}{HTML}{47c2bb}
\definecolor{remark-bg-color}{HTML}{e3f5f4}

\definecolor{code-bg-color}{HTML}{f3f4f6} % todo: temp color
\definecolor{code-border-color}{HTML}{dadce0} % todo: temp color

\definecolor{hl-yellow-color}{HTML}{fef3c7}


% -------------------- Macros --------------------
% highlight function
\newcommand{\highlight}[2]{%
    \begingroup
    \sethlcolor{#1}%
    \hl{#2}%
    \endgroup
}

\newcommand{\hlYellow}[1]{%
    \highlight{hl-yellow-color}{#1}%
}


% inlineCode (without border)
\newcommand{\inlineCodeWithoutBorder}[1]{%
        {\small\tt \highlight{code-bg-color}{#1}}%
}


% inlineCode (with border)
\usepackage[most]{tcolorbox}
\tcbset{
    on line,
    boxsep=2px,
    left=0pt,
    right=0pt,
    top=0pt,
    bottom=0pt,
    boxrule=0.5px,
    colframe=code-border-color,
    colback=code-bg-color,
    highlight math style={enhanced},
    breakable
}

\newcommand{\inlineCode}[1]{%
    \tcbox{{\small\tt #1}}%
}

% minted background color
\mdfdefinestyle{code-style}{%
    innertopmargin=10px,
    innerbottommargin=10px,
    linecolor=definition,
    backgroundcolor=definition-bg,
    linewidth=0px,
}
\newmdenv[style=code-style]{code}

\newcommand\clovisCode[1]{
    \begin{code}
        #1
    \end{code}
}


% -------------------- colorful-blocks --------------------
\mdfdefinestyle{definition-style}{%
    innertopmargin=10px,
    innerbottommargin=10px,
    linecolor=definition,
    backgroundcolor=definition-bg,
    linewidth=2px,
    topline=false,
    bottomline=false,
    rightline=false,
}
\newmdenv[style=definition-style]{definition}

\newcommand\clovisDefinition[2]{
    \begin{definition}
    { \scriptsize \textcolor{definition}{\faIcon{graduation-cap} \textbf{DEFINITION}}}
        \vspace{3px}
        \\ \underline{\textbf{#1}}
        \vspace{2.5px}
        \\ #2
    \end{definition}
}


\newcommand\clovisColorfulBlock[3]{
% #1 = colorful block name (danger)
% #2 = colorful block name, with the first letter in uppercase (Danger)
% #3 = icon name (exclamation-triangle)

    \mdfdefinestyle{#1-style}{%
        innertopmargin=10px,
        innerbottommargin=10px,
        innerrightmargin=12px, % compensate for lack of line here (default 10px + 2px for linewidth)
        linecolor=#1-color,
        backgroundcolor=#1-bg-color,
        roundcorner=2px,
        linewidth=2px,
        topline=false,
        bottomline=false,
        rightline=false,
    }
    \newmdenv[style=#1-style]{#1}


    \expandafter\newcommand\csname clovis#2\endcsname[1]{
        \begin{#1}
        {\scriptsize \textcolor{#1-color}{\faIcon{#3} \textbf{\MakeUppercase{#1}}}}
            \vspace{3px}
            \\ ##1
        \end{#1}
    }
}

\clovisColorfulBlock{example}{Example}{graduation-cap}
\clovisColorfulBlock{byheart}{Byheart}{graduation-cap}
\clovisColorfulBlock{danger}{Danger}{exclamation-triangle}
\clovisColorfulBlock{summary}{Summary}{graduation-cap}
\clovisColorfulBlock{reminder}{Reminder}{graduation-cap}
\clovisColorfulBlock{advice}{Advice}{graduation-cap}
\clovisColorfulBlock{remark}{Remark}{graduation-cap}


% -------------------- Quotes & Excerpts --------------------
\mdfdefinestyle{quote-style}{%
    innertopmargin=6px,
    innerbottommargin=6px,
    innerleftmargin=2px,
    linecolor=gray300,
    backgroundcolor=white,
    linewidth=1px,
    topline=false,
    bottomline=false,
    rightline=false,
}
\newmdenv[style=quote-style]{clovis-quote}

\newcommand\clovisQuote[4]{
    \begin{clovis-quote}
        \begin{tabular}{ c p{310px} }
        {\large \textbf{``}}
            \hspace{-10px} & {\small \textbf{#1}} {\large \textbf{''}} \vspace{4px} \\
            \notblank{#2}{ & - {\small \textcolor{gray700}{#2}}}{} % \textsc{#2} can be nice (ONLY with serif font)
            \notblank{#3}{
                \notblank{#4}{
                    \vspace{1px} \\ & {\small \textcolor{gray700}{\textit{#3}, #4}}
                }{
                    \vspace{1px} \\ & {\small \textcolor{gray700}{\textit{#3}}}
                }
            }{
                \notblank{#4}{
                        {\hspace{-13px} \small \textcolor{gray700}{, #4}}
                }{
                }
            }
        \end{tabular}
        %{\textbf{``  #1  ''}}
        %\vspace{4px}
    \end{clovis-quote}
}


% -------------------- Document --------------------
\title{Clovis LaTeX Package}
\author{The Clovis Team}
\date{2022}


% Table of contents for h4 %todo: improve support for h4 titles
\setcounter{tocdepth}{4}
\setcounter{secnumdepth}{4}

\begin{document}
    \color{gray800}
    \normalsize
    \maketitle

    \tableofcontents
    \newpage


    \section{Introduction}

    salut $5x + 3$
    \clovisQuote{Salut tout le monde}{Anne}{Conversation Discord}{2021}
    \clovisQuote{Salut tout le monde}{Anne}{Conversation Discord}{}
    \clovisQuote{Salut tout le monde}{Anne}{}{2021}
    \clovisQuote{Salut tout le monde}{Anne}{}{}
    \clovisQuote{Salut tout le monde}{}{}{}
    \clovisQuote{Salut tout le monde}{}{Conversation Discord}{2021}
    \clovisQuote{Salut tout le monde}{}{Conversation Discord}{}
    \clovisQuote{Salut tout le monde}{}{}{2021}
    \clovisQuote{The word policy is commonly used to designate the most important choices made either in organised or private life. Policy is free of many of the undesirable connotations clustered around the word "political" which is often believed to imply partisanship or corruption.}{Harold Dwight Lasswell}{}{}
    \clovisQuote{The word policy is commonly used to designate the most important choices made either in organised or private life. Policy is free of many of the undesirable connotations clustered around the word "political" which is often believed to imply partisanship or corruption. The word policy is commonly used to designate the most important choices made either in organised or private life. Policy is free of many of the undesirable connotations clustered around the word "political" which is often believed to imply partisanship or corruption.}{Harold Dwight Lasswell}{}{}

    \clovisDefinition{Title}{Definition}

    \clovisDefinition{Title}{
        Definition
    }

    \clovisDanger{
        $5x+3$ is a math expression.
    }

    \clovisDanger{$\left|  \begin{matrix}
                               v_n \leqslant u_n \leqslant w_n \\ \\ \displaystyle\lim_{n\to +\infty}v_n=\lim_{n\to +\infty}w_n=\ell
    \end{matrix}  \right\}  \displaystyle\lim_{n\to +\infty}u_n=\ell$}

    \clovisDanger{\(\left|  \begin{matrix}
                                v_n \leqslant u_n \leqslant w_n \\ \\ \displaystyle\lim_{n\to +\infty}v_n=\lim_{n\to +\infty}w_n=\ell
    \end{matrix}  \right\}  \displaystyle\lim_{n\to +\infty}u_n=\ell\)}

    \clovisDanger{$$\left|  \begin{matrix}
                                v_n \leqslant u_n \leqslant w_n \\ \\ \displaystyle\lim_{n\to +\infty}v_n=\lim_{n\to +\infty}w_n=\ell
    \end{matrix}  \right\}  \displaystyle\lim_{n\to +\infty}u_n=\ell$$}

    \clovisDanger{\[\left|  \begin{matrix}
                                v_n \leqslant u_n \leqslant w_n \\ \\ \displaystyle\lim_{n\to +\infty}v_n=\lim_{n\to +\infty}w_n=\ell
    \end{matrix}  \right\}  \displaystyle\lim_{n\to +\infty}u_n=\ell\]}

    \clovisDanger{$\begin{array}{|c|c|c|c|c|c|c|}
                       \hline\lim_{n\to +\infty}u_n & L & L & L & +\infty & -\infty & +\infty \\ \hline \\\displaystyle\lim_{n\to +\infty}v_n & L' & +\infty & - \infty & +\infty & -\infty & -\infty \\\\ \hline \\\displaystyle\lim_{n\to +\infty}(u_n+v_n) & L+L' & +\infty & -\infty & +\infty & -\infty & {\text{F.I.}} \\\\\hline
    \end{array}$}
%

    \begin{minted}[
        bgcolor=code-bg-color,
        linenos
    ]{java}
// nom de classe = nom du fichier
class HelloWorld {
        public static void main(String[] args) {
                System.out.println("Hello world");
        }
}
    \end{minted}

    \begin{mdframed}[backgroundcolor=code-bg-color, linewidth=0]
        \begin{minted}[linenos, xleftmargin=12px]{java}
// nom de classe = nom du fichier
class HelloWorld {
        public static void main(String[] args) {
                System.out.println("Hello world");
        }
}
        \end{minted}
    \end{mdframed}

\end{document}
